\documentclass[]{article}
\usepackage{lmodern}
\usepackage{amssymb,amsmath}
\usepackage{ifxetex,ifluatex}
\usepackage{fixltx2e} % provides \textsubscript
\ifnum 0\ifxetex 1\fi\ifluatex 1\fi=0 % if pdftex
  \usepackage[T1]{fontenc}
  \usepackage[utf8]{inputenc}
\else % if luatex or xelatex
  \ifxetex
    \usepackage{mathspec}
  \else
    \usepackage{fontspec}
  \fi
  \defaultfontfeatures{Ligatures=TeX,Scale=MatchLowercase}
\fi
% use upquote if available, for straight quotes in verbatim environments
\IfFileExists{upquote.sty}{\usepackage{upquote}}{}
% use microtype if available
\IfFileExists{microtype.sty}{%
\usepackage{microtype}
\UseMicrotypeSet[protrusion]{basicmath} % disable protrusion for tt fonts
}{}
\usepackage[margin=1in]{geometry}
\usepackage{hyperref}
\hypersetup{unicode=true,
            pdfborder={0 0 0},
            breaklinks=true}
\urlstyle{same}  % don't use monospace font for urls
\usepackage{color}
\usepackage{fancyvrb}
\newcommand{\VerbBar}{|}
\newcommand{\VERB}{\Verb[commandchars=\\\{\}]}
\DefineVerbatimEnvironment{Highlighting}{Verbatim}{commandchars=\\\{\}}
% Add ',fontsize=\small' for more characters per line
\usepackage{framed}
\definecolor{shadecolor}{RGB}{248,248,248}
\newenvironment{Shaded}{\begin{snugshade}}{\end{snugshade}}
\newcommand{\KeywordTok}[1]{\textcolor[rgb]{0.13,0.29,0.53}{\textbf{#1}}}
\newcommand{\DataTypeTok}[1]{\textcolor[rgb]{0.13,0.29,0.53}{#1}}
\newcommand{\DecValTok}[1]{\textcolor[rgb]{0.00,0.00,0.81}{#1}}
\newcommand{\BaseNTok}[1]{\textcolor[rgb]{0.00,0.00,0.81}{#1}}
\newcommand{\FloatTok}[1]{\textcolor[rgb]{0.00,0.00,0.81}{#1}}
\newcommand{\ConstantTok}[1]{\textcolor[rgb]{0.00,0.00,0.00}{#1}}
\newcommand{\CharTok}[1]{\textcolor[rgb]{0.31,0.60,0.02}{#1}}
\newcommand{\SpecialCharTok}[1]{\textcolor[rgb]{0.00,0.00,0.00}{#1}}
\newcommand{\StringTok}[1]{\textcolor[rgb]{0.31,0.60,0.02}{#1}}
\newcommand{\VerbatimStringTok}[1]{\textcolor[rgb]{0.31,0.60,0.02}{#1}}
\newcommand{\SpecialStringTok}[1]{\textcolor[rgb]{0.31,0.60,0.02}{#1}}
\newcommand{\ImportTok}[1]{#1}
\newcommand{\CommentTok}[1]{\textcolor[rgb]{0.56,0.35,0.01}{\textit{#1}}}
\newcommand{\DocumentationTok}[1]{\textcolor[rgb]{0.56,0.35,0.01}{\textbf{\textit{#1}}}}
\newcommand{\AnnotationTok}[1]{\textcolor[rgb]{0.56,0.35,0.01}{\textbf{\textit{#1}}}}
\newcommand{\CommentVarTok}[1]{\textcolor[rgb]{0.56,0.35,0.01}{\textbf{\textit{#1}}}}
\newcommand{\OtherTok}[1]{\textcolor[rgb]{0.56,0.35,0.01}{#1}}
\newcommand{\FunctionTok}[1]{\textcolor[rgb]{0.00,0.00,0.00}{#1}}
\newcommand{\VariableTok}[1]{\textcolor[rgb]{0.00,0.00,0.00}{#1}}
\newcommand{\ControlFlowTok}[1]{\textcolor[rgb]{0.13,0.29,0.53}{\textbf{#1}}}
\newcommand{\OperatorTok}[1]{\textcolor[rgb]{0.81,0.36,0.00}{\textbf{#1}}}
\newcommand{\BuiltInTok}[1]{#1}
\newcommand{\ExtensionTok}[1]{#1}
\newcommand{\PreprocessorTok}[1]{\textcolor[rgb]{0.56,0.35,0.01}{\textit{#1}}}
\newcommand{\AttributeTok}[1]{\textcolor[rgb]{0.77,0.63,0.00}{#1}}
\newcommand{\RegionMarkerTok}[1]{#1}
\newcommand{\InformationTok}[1]{\textcolor[rgb]{0.56,0.35,0.01}{\textbf{\textit{#1}}}}
\newcommand{\WarningTok}[1]{\textcolor[rgb]{0.56,0.35,0.01}{\textbf{\textit{#1}}}}
\newcommand{\AlertTok}[1]{\textcolor[rgb]{0.94,0.16,0.16}{#1}}
\newcommand{\ErrorTok}[1]{\textcolor[rgb]{0.64,0.00,0.00}{\textbf{#1}}}
\newcommand{\NormalTok}[1]{#1}
\usepackage{longtable,booktabs}
\usepackage{graphicx,grffile}
\makeatletter
\def\maxwidth{\ifdim\Gin@nat@width>\linewidth\linewidth\else\Gin@nat@width\fi}
\def\maxheight{\ifdim\Gin@nat@height>\textheight\textheight\else\Gin@nat@height\fi}
\makeatother
% Scale images if necessary, so that they will not overflow the page
% margins by default, and it is still possible to overwrite the defaults
% using explicit options in \includegraphics[width, height, ...]{}
\setkeys{Gin}{width=\maxwidth,height=\maxheight,keepaspectratio}
\IfFileExists{parskip.sty}{%
\usepackage{parskip}
}{% else
\setlength{\parindent}{0pt}
\setlength{\parskip}{6pt plus 2pt minus 1pt}
}
\setlength{\emergencystretch}{3em}  % prevent overfull lines
\providecommand{\tightlist}{%
  \setlength{\itemsep}{0pt}\setlength{\parskip}{0pt}}
\setcounter{secnumdepth}{0}
% Redefines (sub)paragraphs to behave more like sections
\ifx\paragraph\undefined\else
\let\oldparagraph\paragraph
\renewcommand{\paragraph}[1]{\oldparagraph{#1}\mbox{}}
\fi
\ifx\subparagraph\undefined\else
\let\oldsubparagraph\subparagraph
\renewcommand{\subparagraph}[1]{\oldsubparagraph{#1}\mbox{}}
\fi

%%% Use protect on footnotes to avoid problems with footnotes in titles
\let\rmarkdownfootnote\footnote%
\def\footnote{\protect\rmarkdownfootnote}

%%% Change title format to be more compact
\usepackage{titling}

% Create subtitle command for use in maketitle
\newcommand{\subtitle}[1]{
  \posttitle{
    \begin{center}\large#1\end{center}
    }
}

\setlength{\droptitle}{-2em}

  \title{}
    \pretitle{\vspace{\droptitle}}
  \posttitle{}
    \author{}
    \preauthor{}\postauthor{}
    \date{}
    \predate{}\postdate{}
  

\begin{document}

Complete all \textbf{Exercises}, and submit answers to
\textbf{Questions} in the \textbf{Quiz: Week 3 Lab} on Coursera.

\subsection{Getting Started}\label{getting-started}

In this lab we will review exploratory data analysis using the
\texttt{ggplot2} package for data visualization, which is included in
the \texttt{tidyverse}. The main focus of this lab is to be able to
obtain and interpret credible intervals and hypothesis tests using
Bayesian methods for numerical variables. The data and functions for
inference can be found in the companion package for this course,
\texttt{statsr}.

Let's load the necessary packages for this week's lab:

\begin{Shaded}
\begin{Highlighting}[]
\KeywordTok{library}\NormalTok{(PairedData)}
\KeywordTok{library}\NormalTok{(tidyverse)}
\KeywordTok{library}\NormalTok{(statsr)}
\end{Highlighting}
\end{Shaded}

\subsubsection{The data}\label{the-data}

In 2004, the state of North Carolina released a large data set
containing information on births recorded in this state. This data set
is useful to researchers studying the relation between habits and
practices of expectant mothers and the birth of their children. We will
work with a random sample of observations from this data set.

Let's load the \texttt{nc} data set into our workspace from the
\texttt{statsr} package.

\begin{Shaded}
\begin{Highlighting}[]
\KeywordTok{data}\NormalTok{(nc)}
\end{Highlighting}
\end{Shaded}

We have observations on 13 different variables, some categorical and
some numerical. The meaning of each variable is as follows.

\begin{longtable}[]{@{}ll@{}}
\toprule
\begin{minipage}[b]{0.22\columnwidth}\raggedright\strut
variable\strut
\end{minipage} & \begin{minipage}[b]{0.60\columnwidth}\raggedright\strut
description\strut
\end{minipage}\tabularnewline
\midrule
\endhead
\begin{minipage}[t]{0.22\columnwidth}\raggedright\strut
\texttt{fage}\strut
\end{minipage} & \begin{minipage}[t]{0.60\columnwidth}\raggedright\strut
father's age in years.\strut
\end{minipage}\tabularnewline
\begin{minipage}[t]{0.22\columnwidth}\raggedright\strut
\texttt{mage}\strut
\end{minipage} & \begin{minipage}[t]{0.60\columnwidth}\raggedright\strut
mother's age in years.\strut
\end{minipage}\tabularnewline
\begin{minipage}[t]{0.22\columnwidth}\raggedright\strut
\texttt{mature}\strut
\end{minipage} & \begin{minipage}[t]{0.60\columnwidth}\raggedright\strut
maturity status of mother.\strut
\end{minipage}\tabularnewline
\begin{minipage}[t]{0.22\columnwidth}\raggedright\strut
\texttt{weeks}\strut
\end{minipage} & \begin{minipage}[t]{0.60\columnwidth}\raggedright\strut
length of pregnancy in weeks.\strut
\end{minipage}\tabularnewline
\begin{minipage}[t]{0.22\columnwidth}\raggedright\strut
\texttt{premie}\strut
\end{minipage} & \begin{minipage}[t]{0.60\columnwidth}\raggedright\strut
whether the birth was classified as premature (premie) or
full-term.\strut
\end{minipage}\tabularnewline
\begin{minipage}[t]{0.22\columnwidth}\raggedright\strut
\texttt{visits}\strut
\end{minipage} & \begin{minipage}[t]{0.60\columnwidth}\raggedright\strut
number of hospital visits during pregnancy.\strut
\end{minipage}\tabularnewline
\begin{minipage}[t]{0.22\columnwidth}\raggedright\strut
\texttt{marital}\strut
\end{minipage} & \begin{minipage}[t]{0.60\columnwidth}\raggedright\strut
whether mother is \texttt{married} or \texttt{not\ married} at
birth.\strut
\end{minipage}\tabularnewline
\begin{minipage}[t]{0.22\columnwidth}\raggedright\strut
\texttt{gained}\strut
\end{minipage} & \begin{minipage}[t]{0.60\columnwidth}\raggedright\strut
weight gained by mother during pregnancy in pounds.\strut
\end{minipage}\tabularnewline
\begin{minipage}[t]{0.22\columnwidth}\raggedright\strut
\texttt{weight}\strut
\end{minipage} & \begin{minipage}[t]{0.60\columnwidth}\raggedright\strut
weight of the baby at birth in pounds.\strut
\end{minipage}\tabularnewline
\begin{minipage}[t]{0.22\columnwidth}\raggedright\strut
\texttt{lowbirthweight}\strut
\end{minipage} & \begin{minipage}[t]{0.60\columnwidth}\raggedright\strut
whether baby was classified as low birthweight (\texttt{low}) or not
(\texttt{not\ low}).\strut
\end{minipage}\tabularnewline
\begin{minipage}[t]{0.22\columnwidth}\raggedright\strut
\texttt{gender}\strut
\end{minipage} & \begin{minipage}[t]{0.60\columnwidth}\raggedright\strut
gender of the baby, \texttt{female} or \texttt{male}.\strut
\end{minipage}\tabularnewline
\begin{minipage}[t]{0.22\columnwidth}\raggedright\strut
\texttt{habit}\strut
\end{minipage} & \begin{minipage}[t]{0.60\columnwidth}\raggedright\strut
status of the mother as a \texttt{nonsmoker} or a \texttt{smoker}.\strut
\end{minipage}\tabularnewline
\begin{minipage}[t]{0.22\columnwidth}\raggedright\strut
\texttt{whitemom}\strut
\end{minipage} & \begin{minipage}[t]{0.60\columnwidth}\raggedright\strut
whether mom is \texttt{white} or \texttt{not\ white}.\strut
\end{minipage}\tabularnewline
\bottomrule
\end{longtable}

Note: These data should be familiar for those who took the
\emph{Inferential Statistics} course as part of the \emph{Statistics
with R} specialization, where the \texttt{nc} data were used in the
\textbf{Inference for Numerical Data} lab.

\subsubsection{EDA}\label{eda}

As a first step in the analysis, let's take a look at the variables in
the dataset and how \texttt{R} has encoded them. The most straight
forward way of doing this is using the \texttt{glimpse} function.

\begin{Shaded}
\begin{Highlighting}[]
\KeywordTok{glimpse}\NormalTok{(nc)}
\end{Highlighting}
\end{Shaded}

\begin{verbatim}
## Observations: 1,000
## Variables: 13
## $ fage           <int> NA, NA, 19, 21, NA, NA, 18, 17, NA, 20, 30, NA,...
## $ mage           <int> 13, 14, 15, 15, 15, 15, 15, 15, 16, 16, 16, 16,...
## $ mature         <fct> younger mom, younger mom, younger mom, younger ...
## $ weeks          <int> 39, 42, 37, 41, 39, 38, 37, 35, 38, 37, 45, 42,...
## $ premie         <fct> full term, full term, full term, full term, ful...
## $ visits         <int> 10, 15, 11, 6, 9, 19, 12, 5, 9, 13, 9, 8, 4, 12...
## $ marital        <fct> married, married, married, married, married, ma...
## $ gained         <int> 38, 20, 38, 34, 27, 22, 76, 15, NA, 52, 28, 34,...
## $ weight         <dbl> 7.63, 7.88, 6.63, 8.00, 6.38, 5.38, 8.44, 4.69,...
## $ lowbirthweight <fct> not low, not low, not low, not low, not low, lo...
## $ gender         <fct> male, male, female, male, female, male, male, m...
## $ habit          <fct> nonsmoker, nonsmoker, nonsmoker, nonsmoker, non...
## $ whitemom       <fct> not white, not white, white, white, not white, ...
\end{verbatim}

Another useful function is \texttt{summary} which provides the range,
quartiles, and means for numeric variables and counts for categorical
variables. Additionally, if there are any missing observations (denoted
\texttt{NA}), summary will provide the number of missing cases for each
variable. Note that the output of the summary function can be too long
and difficult to parse visually and interpret if the dataset has a large
number of variables.

\begin{Shaded}
\begin{Highlighting}[]
\KeywordTok{summary}\NormalTok{(nc)}
\end{Highlighting}
\end{Shaded}

\begin{verbatim}
##       fage            mage            mature        weeks      
##  Min.   :14.00   Min.   :13   mature mom :133   Min.   :20.00  
##  1st Qu.:25.00   1st Qu.:22   younger mom:867   1st Qu.:37.00  
##  Median :30.00   Median :27                     Median :39.00  
##  Mean   :30.26   Mean   :27                     Mean   :38.33  
##  3rd Qu.:35.00   3rd Qu.:32                     3rd Qu.:40.00  
##  Max.   :55.00   Max.   :50                     Max.   :45.00  
##  NA's   :171                                    NA's   :2      
##        premie        visits            marital        gained     
##  full term:846   Min.   : 0.0   married    :386   Min.   : 0.00  
##  premie   :152   1st Qu.:10.0   not married:613   1st Qu.:20.00  
##  NA's     :  2   Median :12.0   NA's       :  1   Median :30.00  
##                  Mean   :12.1                     Mean   :30.33  
##                  3rd Qu.:15.0                     3rd Qu.:38.00  
##                  Max.   :30.0                     Max.   :85.00  
##                  NA's   :9                        NA's   :27     
##      weight       lowbirthweight    gender          habit    
##  Min.   : 1.000   low    :111    female:503   nonsmoker:873  
##  1st Qu.: 6.380   not low:889    male  :497   smoker   :126  
##  Median : 7.310                               NA's     :  1  
##  Mean   : 7.101                                              
##  3rd Qu.: 8.060                                              
##  Max.   :11.750                                              
##                                                              
##       whitemom  
##  not white:284  
##  white    :714  
##  NA's     :  2  
##                 
##                 
##                 
## 
\end{verbatim}

As you review the variable summaries, consider which variables are
categorical and which are numerical. For numerical variables, are there
outliers? If you aren't sure or want to take a closer look at the data,
you can make a graph.

For example, we can examine the distribution of the amount of weight
that a mother \texttt{gained} with a histogram.

\begin{Shaded}
\begin{Highlighting}[]
\KeywordTok{ggplot}\NormalTok{(}\DataTypeTok{data =}\NormalTok{ nc, }\KeywordTok{aes}\NormalTok{(}\DataTypeTok{x =}\NormalTok{ gained)) }\OperatorTok{+}
\StringTok{  }\KeywordTok{geom_histogram}\NormalTok{(}\DataTypeTok{binwidth =} \DecValTok{5}\NormalTok{)}
\end{Highlighting}
\end{Shaded}

\begin{verbatim}
## Warning: Removed 27 rows containing non-finite values (stat_bin).
\end{verbatim}

\includegraphics{Week3Lab_files/figure-latex/hist-weight-1.pdf}

This function says to plot the \texttt{gained} variable from the
\texttt{nc} data frame on the x-axis. It also defines a \texttt{geom}
(short for geometric object), which describes the type of plot you will
produce. We used a binwidth of 5, however you can change this value and
see how it affects the shape of the histogram. Also note that the
function results in a warning saying that 27 rows have been removed.
This is because 27 observations in the data have \texttt{NA} values for
weight gained. You can confirm this by peeking back at the summary
output above. If you need a refresher on using \texttt{ggplot2}, you may
want to take some time to review the material in the earlier courses in
this specialization.

How many of the 13 variables are categorical?

\begin{itemize}
\tightlist
\item
  5
\item
  6
\item
  7
\item
  8
\end{itemize}

\begin{Shaded}
\begin{Highlighting}[]
\CommentTok{# Type your code for Question 1 here.}
\end{Highlighting}
\end{Shaded}

We will start with analyzing the weight of the babies at birth, which is
contained in the variable \texttt{weight}.

Use a visualization such as a histogram and summary statistics tools in
R to analyze the distribution of \texttt{weight}. Which of the following
best describes the distribution of \texttt{weight}?

\begin{itemize}
\tightlist
\item
  Left skewed
\item
  Right skewed
\item
  Uniformly distributed
\item
  Normally distributed
\end{itemize}

\begin{Shaded}
\begin{Highlighting}[]
\CommentTok{# Type your code for Question 2 here.}
\end{Highlighting}
\end{Shaded}

The variable \texttt{premie} in the dataframe classifies births on
whether they were full-term or premie. We can use some of the functions
of \texttt{dplyr} to create a new dataframe to limit the analysis to
full term births.

\begin{Shaded}
\begin{Highlighting}[]
\NormalTok{nc_fullterm =}\StringTok{ }\KeywordTok{filter}\NormalTok{(nc, premie }\OperatorTok{==}\StringTok{ 'full term'}\NormalTok{)}
\KeywordTok{summary}\NormalTok{(nc_fullterm)}
\end{Highlighting}
\end{Shaded}

\begin{verbatim}
##       fage            mage            mature        weeks      
##  Min.   :14.00   Min.   :13   mature mom :109   Min.   :37.00  
##  1st Qu.:25.00   1st Qu.:22   younger mom:737   1st Qu.:38.00  
##  Median :30.00   Median :27                     Median :39.00  
##  Mean   :30.24   Mean   :27                     Mean   :39.25  
##  3rd Qu.:35.00   3rd Qu.:32                     3rd Qu.:40.00  
##  Max.   :50.00   Max.   :50                     Max.   :45.00  
##  NA's   :132                                                   
##        premie        visits             marital        gained     
##  full term:846   Min.   : 0.00   married    :312   Min.   : 0.00  
##  premie   :  0   1st Qu.:10.00   not married:534   1st Qu.:22.00  
##                  Median :12.00                     Median :30.00  
##                  Mean   :12.35                     Mean   :31.13  
##                  3rd Qu.:15.00                     3rd Qu.:40.00  
##                  Max.   :30.00                     Max.   :85.00  
##                  NA's   :6                         NA's   :19     
##      weight       lowbirthweight    gender          habit    
##  Min.   : 3.750   low    : 30    female:431   nonsmoker:739  
##  1st Qu.: 6.750   not low:816    male  :415   smoker   :107  
##  Median : 7.440                                              
##  Mean   : 7.459                                              
##  3rd Qu.: 8.190                                              
##  Max.   :11.750                                              
##                                                              
##       whitemom  
##  not white:228  
##  white    :616  
##  NA's     :  2  
##                 
##                 
##                 
## 
\end{verbatim}

The \texttt{filter} function selects variables all variables from the
dataframe \texttt{nc} where the condition \texttt{premie} equals ``full
term'' is met.

Repeat the visualization and summary with the weights from full term
term births. Does \texttt{weight} appear to be approximately normally
distributed?

\subsection{Inference}\label{inference}

Next we will introduce a function \texttt{bayes\_inference} that we will
use for constructing credible intervals and conducting hypothesis tests.
The following illustrates how we would use the function
\texttt{bayes\_inference} to construct a 95\% credible interval of
\texttt{weight}; the Bayesian analogue to a 95\% confidence interval.

\begin{Shaded}
\begin{Highlighting}[]
\KeywordTok{bayes_inference}\NormalTok{(}\DataTypeTok{y =}\NormalTok{ weight, }\DataTypeTok{data =}\NormalTok{ nc_fullterm, }
                \DataTypeTok{statistic =} \StringTok{"mean"}\NormalTok{, }\DataTypeTok{type =} \StringTok{"ci"}\NormalTok{,  }
                \DataTypeTok{prior_family =} \StringTok{"JZS"}\NormalTok{, }\DataTypeTok{mu_0 =} \FloatTok{7.7}\NormalTok{, }\DataTypeTok{rscale =} \DecValTok{1}\NormalTok{,}
                \DataTypeTok{method =} \StringTok{"simulation"}\NormalTok{,}
                \DataTypeTok{cred_level =} \FloatTok{0.95}\NormalTok{)}
\end{Highlighting}
\end{Shaded}

\begin{verbatim}
## Single numerical variable
## n = 846, y-bar = 7.4594, s = 1.075
## (Assuming Zellner-Siow Cauchy prior:  mu | sigma^2 ~ C(7.7, 1*sigma)
## (Assuming improper Jeffreys prior: p(sigma^2) = 1/sigma^2
## 
## Posterior Summaries
##            2.5%       25%       50%       75%      97.5%
## mu    7.3856447  7.433359  7.458152  7.484357   7.531028
## sigma 1.0274558  1.058485  1.075886  1.093907   1.129180
## n_0   0.9728835 10.051580 24.343128 47.987231 126.158908
## 
## 95% CI for mu: (7.3856, 7.531)
\end{verbatim}

\begin{center}\includegraphics[width=0.7\linewidth]{Week3Lab_files/figure-latex/mean-inference-1} \end{center}

Let's look at the meanings of the arguments of this custom function. The
first argument \texttt{y} specifies the response variable that we are
interested in: \texttt{weight}. The second argument, \texttt{data},
specifies the dataset \texttt{nc\_fullterm} that contains the variable
\texttt{weight}. The third argument \texttt{statistic} is the sample
statistic we're using, or similarly, the population parameter we're
estimating. The argument \texttt{type} specifies the type of inference
that we want: credible intervals (\texttt{type\ =\ "ci"}), or hypothesis
tests (\texttt{type\ =\ "ht"}). The argument \texttt{prior} indicates
which prior distribution for any unknown parameters we will use for
inference or testing, with options \texttt{JZS} (the
Jeffreys-Zellner-Siow prior which is the Jeffreys prior for the unknown
variance and a Cauchy prior for the mean), \texttt{JUI} (the
Jeffreys-Unit Information prior which is the Jeffreys prior for the
variance and the Unit Information Gaussian prior for the mean),
\texttt{NG} (the conjugate Normal-Gamma prior for the mean and inverse
of the variance) or \texttt{ref} (the independent Jeffreys reference
prior for the variance and the uniform prior for the mean). As we would
like to use the same prior for constructing credible intervals and
hypothesis testing later with results that are robust if we mis-specify
the prior, we will use the \texttt{JZS} option. For all of the
\texttt{prior\_family} options, we need to specify prior
hyperparameters. For \texttt{JZS}, the prior on standardized effect
\(\mu/\sigma\) is a Cauchy centered at \texttt{mu\_0} and with a scale
of \texttt{rscale}. By default these are zero and one respectively. The
average birthweight for full term births in the US in 1994-1996 was 7.7
pounds, which we will use as the center of our prior distribution using
the argument \texttt{mu\_0\ =\ 7.7}. We will use the default argument
\texttt{rscale\ =\ 1}. The method of inference can be either
\texttt{method\ =\ "theoretical"} (theoretical based) or
\texttt{"simulation"} based; in the case of the \texttt{JZS} prior for
credible intervals, \texttt{"simulation"} is the only option as there
are no closed form results. We also specify that we are looking for the
95\% credible interval by setting \texttt{cred\_level\ =\ 0.95}, which
is the default. For more information on the \texttt{bayes\_inference}
function see the help file with \texttt{?bayes\_inference}.

Which of the following corresponds to the \textbf{95\%} credible
interval for the average birth weight of all full-term babies born in
North Carolina?

\begin{itemize}
\tightlist
\item
  There is a 95\% chance that babies weigh 7.4 to 7.5 pounds.
\item
  There is a 95\% chance that the average weights of babies in this
  sample is between 7.4 an 7.5 pounds.
\item
  There is a 95\% chance that babies on average weigh 7.4 to 7.5 pounds.
\end{itemize}

We can also conduct a Bayesian hypothesis test by calculating Bayes
factors or posterior probabilities. Let us test to see if the average
birth weight in North Carolina for full term births is significantly
different from the US average of 7.7 pounds from 1994-96. The two
competing hypotheses are:

\[ H_1: \mu = 7.7 \] \[ H_2: \mu \ne 7.7 \]

To conduct this hypothesis test, we will need to change the
\texttt{type} argument to hypothesis test, \texttt{type\ =\ "ht"} in the
\texttt{bayes\_inference} function. In addition, we will need to add the
type of alternative hypothesis as an additional argument
\texttt{alternative\ =\ "twosided"}. For faster calculation, change the
method to \texttt{theoretical} and add \texttt{show\_plot=FALSE}.

Based of Jeffrey's scale for interpretation of a Bayes factors how
should we describe the evidence against \(H_1\) from your results for
the hypothesis test?

\begin{itemize}
\tightlist
\item
  Not worth a bare mention
\item
  Positive
\item
  Strong
\item
  Very Strong
\end{itemize}

\begin{Shaded}
\begin{Highlighting}[]
\CommentTok{# Type your code for the Exercise here.}
\end{Highlighting}
\end{Shaded}

\subsection{Prediction using MCMC}\label{prediction-using-mcmc}

A key advantage of Bayesian statistics is predictions and the
probabilistic interpretation of predictions. Much of Bayesian prediction
is done using simulation techniques, some of which was discussed near
the end of this module. We will go over a simple simulation example to
obtain the predictive distribution of the variable \texttt{weight} using
the output of \texttt{bayes\_inference} which we will save to the object
\texttt{weight\_post}:

\begin{Shaded}
\begin{Highlighting}[]
\NormalTok{weight_post =}\StringTok{ }\KeywordTok{bayes_inference}\NormalTok{(}\DataTypeTok{y =}\NormalTok{ weight, }\DataTypeTok{data =}\NormalTok{ nc_fullterm, }
                              \DataTypeTok{statistic =} \StringTok{"mean"}\NormalTok{, }\DataTypeTok{type =} \StringTok{"ci"}\NormalTok{,  }
                              \DataTypeTok{prior_family =} \StringTok{"JZS"}\NormalTok{, }\DataTypeTok{mu_0 =} \FloatTok{7.7}\NormalTok{, }\DataTypeTok{rscale =} \DecValTok{1}\NormalTok{,}
                              \DataTypeTok{method =} \StringTok{"simulation"}\NormalTok{,}
                              \DataTypeTok{cred_level =} \FloatTok{0.95}\NormalTok{)}
\end{Highlighting}
\end{Shaded}

\begin{verbatim}
## Single numerical variable
## n = 846, y-bar = 7.4594, s = 1.075
## (Assuming Zellner-Siow Cauchy prior:  mu | sigma^2 ~ C(7.7, 1*sigma)
## (Assuming improper Jeffreys prior: p(sigma^2) = 1/sigma^2
## 
## Posterior Summaries
##            2.5%      25%       50%       75%      97.5%
## mu    7.3884343 7.434758  7.459568  7.483832   7.531574
## sigma 1.0261542 1.057962  1.075498  1.093574   1.128353
## n_0   0.8174496 9.686676 23.526970 46.876835 124.902867
## 
## 95% CI for mu: (7.3884, 7.5316)
\end{verbatim}

\begin{center}\includegraphics[width=0.7\linewidth]{Week3Lab_files/figure-latex/weight-inference-1} \end{center}

The \texttt{names} function can list the output or objects that are
stored in the object created by \texttt{bayes\_inference}:

\begin{Shaded}
\begin{Highlighting}[]
\KeywordTok{names}\NormalTok{(weight_post)}
\end{Highlighting}
\end{Shaded}

\begin{verbatim}
## [1] "mu"         "post_den"   "cred_level" "post_mean"  "post_sd"   
## [6] "ci"         "samples"    "summary"    "plot"
\end{verbatim}

In particular, the \texttt{samples} object is a \texttt{matrix} or table
which contain the draws from the MCMC simulation, and includes columns
for \texttt{mu} and \texttt{sig2}, which are posterior samples of the
mean and variance respectively. Let's see how we can use these to make
predictions.

\subsubsection{\texorpdfstring{Posterior predictive distribution of new
observation
\(y_{n+1}\)}{Posterior predictive distribution of new observation y\_\{n+1\}}}\label{posterior-predictive-distribution-of-new-observation-y_n1}

The distribution of any new observation conditional on the mean and
variance is \[N(\mu, \sigma^2)\] and if we knew \(\mu\) and \(\sigma^2\)
we could draw a sample from the distribution of the new observation from
the normal distribution. While we do not know \(\mu\) and \(\sigma^2\)
we the draws of \(\mu\) and \(\sigma^2\) from their posterior
distributions. If we substitute these values into the normal
distribution for \(Y_{n+1}\), we can obtain samples from the predictive
distribution for the birth weight for any new observation \(y_{1001}\).

We'll first convert our \texttt{samples} into a dataframe and then use
\texttt{mutate} the create draws from the predictive distribution using
\texttt{rnorm}:

\begin{Shaded}
\begin{Highlighting}[]
\NormalTok{samples =}\StringTok{ }\KeywordTok{as.data.frame}\NormalTok{(weight_post}\OperatorTok{$}\NormalTok{samples)}
\NormalTok{nsim =}\StringTok{ }\KeywordTok{nrow}\NormalTok{(samples)}
\NormalTok{samples =}\StringTok{ }\KeywordTok{mutate}\NormalTok{(samples, }\DataTypeTok{y_pred =} \KeywordTok{rnorm}\NormalTok{(nsim, mu, }\KeywordTok{sqrt}\NormalTok{(sig2)))}
\end{Highlighting}
\end{Shaded}

We can view an estimate of the predictive distribution, by looking at a
smoothed version of the histogram of the simulated data:

\begin{Shaded}
\begin{Highlighting}[]
\KeywordTok{ggplot}\NormalTok{(}\DataTypeTok{data =}\NormalTok{ samples, }\KeywordTok{aes}\NormalTok{(}\DataTypeTok{x =}\NormalTok{ y_pred)) }\OperatorTok{+}\StringTok{ }
\StringTok{  }\KeywordTok{geom_histogram}\NormalTok{(}\KeywordTok{aes}\NormalTok{(}\DataTypeTok{y =}\NormalTok{ ..density..), }\DataTypeTok{bins =} \DecValTok{100}\NormalTok{) }\OperatorTok{+}
\StringTok{  }\KeywordTok{geom_density}\NormalTok{() }\OperatorTok{+}\StringTok{ }
\StringTok{  }\KeywordTok{xlab}\NormalTok{(}\KeywordTok{expression}\NormalTok{(y[new]))}
\end{Highlighting}
\end{Shaded}

\begin{center}\includegraphics[width=0.7\linewidth]{Week3Lab_files/figure-latex/preddens-1} \end{center}

A 95\% central credible interval for a new observation is the interval
(L, U) where \(P(Y_{new} < L \mid Y) = 0.05/2\) and
\(P(Y_{new} > U \mid Y) = 0.05/2)\). In this case, since the posterior
distribution of \(\mu\) and \(Y_{new}\) are both symmetric, we can set L
to be the 0.025 quantile and U to be the 0.975 quantile. Using the
\texttt{quantile} function \texttt{R} we can find the 0.025 and 0.975,
as well as median (0.50) quantiles of the predictive distribution:

\begin{Shaded}
\begin{Highlighting}[]
\NormalTok{dplyr}\OperatorTok{::}\KeywordTok{select}\NormalTok{(samples, mu, y_pred) }\OperatorTok
\StringTok{  }\KeywordTok{map}\NormalTok{(quantile, }\DataTypeTok{probs=}\KeywordTok{c}\NormalTok{(}\FloatTok{0.025}\NormalTok{, }\FloatTok{0.50}\NormalTok{, }\FloatTok{0.975}\NormalTok{))}
\end{Highlighting}
\end{Shaded}

\begin{verbatim}
## $mu
##     2.5%      50%    97.5% 
## 7.388434 7.459568 7.531574 
## 
## $y_pred
##     2.5%      50%    97.5% 
## 5.337227 7.460254 9.568247
\end{verbatim}

In the above code we are using \texttt{dplyr:select} to select just the
columns \texttt{mu} and \texttt{y\_pred} from \texttt{samples}. The
usage of \texttt{dplyr:} preceeding \texttt{select}, ensures that we are
using the \texttt{select} function from the \texttt{dplyr} package to
avoid possible name conflicts, as several packages have a
\texttt{select} function. We are also taking advantage of the pipe
operator to send the selected columns to the \texttt{map} function to
apply the \texttt{quantile} function to each of the selected columns for
the probabilities in the argument \texttt{probs} to \texttt{quantile}.

For predicting the birth weight of a new full term baby in NC,

\begin{itemize}
\tightlist
\item
  there is a 95\% chance that their birth weight will be 7.4 to 7.5
  pounds.
\item
  there is a 95\% chance that their birth weight will be on average 7.4
  to 7.5 pounds.
\item
  there is a 95\% chance that their birth weight will be 5.4 to 9.5
  pounds.
\item
  there is 50\% chance that their birth weight will be 7.4 pounds.
\end{itemize}

\begin{Shaded}
\begin{Highlighting}[]
\CommentTok{# Type your code for Question 5 here.}
\end{Highlighting}
\end{Shaded}

Repeat the above analysis but find the predictive distribution for
babies that were premature.

\begin{Shaded}
\begin{Highlighting}[]
\CommentTok{# Type your code for Exersice 2 here.}
\end{Highlighting}
\end{Shaded}

\subsection{Bayesian inference for two independent
means}\label{bayesian-inference-for-two-independent-means}

Next, let us consider whether there is a difference of baby weights for
babies born to smokers and non-smokers. Here we will use the variable
\texttt{habit} to distinguish between babies born to mothers who smoked
and babies born to mothers who were non-smokers. Plotting the data is a
useful first step because it helps us quickly visualize trends, identify
strong associations, and develop research questions.

To create side by side boxplots by levels of a categorical variable
\texttt{x}, we can use the following:

\begin{Shaded}
\begin{Highlighting}[]
\KeywordTok{ggplot}\NormalTok{(nc, }\KeywordTok{aes}\NormalTok{(}\DataTypeTok{x =}\NormalTok{ habit, }\DataTypeTok{y =}\NormalTok{ weight)) }\OperatorTok{+}
\StringTok{  }\KeywordTok{geom_boxplot}\NormalTok{()}
\end{Highlighting}
\end{Shaded}

\includegraphics{Week3Lab_files/figure-latex/weight-habit-box-1.pdf}

to create side-by-side boxplots of \texttt{weight} for smokers and
non-smokers.

Construct a side-by-side boxplot of \texttt{habit} and \texttt{weight}
for the data using full term births and compare the two distributions.
Which of the following is \emph{false} about the relationship between
\texttt{habit} and \texttt{weight}?

\begin{itemize}
\tightlist
\item
  Median birth weight of babies born to non-smokers is slightly higher
  than that of babies born to smokers.
\item
  Range of birth weights of female babies are roughly the same as that
  of male babies.
\item
  Both distributions are approximately symmetric.
\item
  The IQRs of the distributions are roughly equal.
\end{itemize}

\begin{Shaded}
\begin{Highlighting}[]
\CommentTok{# Type your code for the question here.}
\end{Highlighting}
\end{Shaded}

The box plots show how the medians of the two distributions compare, but
we can also compare the means of the distributions using the following
to first group the data by the \texttt{habit} variable, and then
calculate the mean \texttt{weight} in these groups using the
\texttt{mean} function, where the \texttt{\%\textgreater{}\%} pipe
operator takes the output of one function and then \emph{pipes} it into
the next function.

\begin{Shaded}
\begin{Highlighting}[]
\NormalTok{nc_fullterm }\OperatorTok
\StringTok{  }\KeywordTok{group_by}\NormalTok{(habit) }\OperatorTok
\StringTok{  }\KeywordTok{summarise}\NormalTok{(}\DataTypeTok{mean_weight =} \KeywordTok{mean}\NormalTok{(weight))}
\end{Highlighting}
\end{Shaded}

\begin{verbatim}
## # A tibble: 2 x 2
##   habit     mean_weight
##   <fct>           <dbl>
## 1 nonsmoker        7.50
## 2 smoker           7.17
\end{verbatim}

There is an observed difference, but is this difference statistically
significant? In order to answer this question we will conduct a Bayesian
hypothesis test.

As before, we can use the \texttt{bayes\_inference} function to test the
hypothesis the mean weight of babies born to non-smokers is different
than the mean weight of babies born to smokers. The call is almost
identical to the single mean case, except now we will provide
\texttt{habit} as an explanatory variable (argument
\texttt{x\ =\ habit}). Here, we use the theoretical method instead of
simulation (argument \texttt{method\ =\ "theoretical"}).

\begin{Shaded}
\begin{Highlighting}[]
\KeywordTok{bayes_inference}\NormalTok{(}\DataTypeTok{y =}\NormalTok{ weight, }\DataTypeTok{x =}\NormalTok{ habit, }\DataTypeTok{data =}\NormalTok{ nc_fullterm, }
                \DataTypeTok{statistic =} \StringTok{"mean"}\NormalTok{, }
                \DataTypeTok{type =} \StringTok{"ht"}\NormalTok{, }\DataTypeTok{alternative =} \StringTok{"twosided"}\NormalTok{, }\DataTypeTok{null =} \DecValTok{0}\NormalTok{, }
                \DataTypeTok{prior =} \StringTok{"JZS"}\NormalTok{, }\DataTypeTok{rscale =} \DecValTok{1}\NormalTok{, }
                \DataTypeTok{method =} \StringTok{"theoretical"}\NormalTok{, }\DataTypeTok{show_plot =} \OtherTok{FALSE}\NormalTok{)}
\end{Highlighting}
\end{Shaded}

\begin{verbatim}
## Response variable: numerical, Explanatory variable: categorical (2 levels)
## n_nonsmoker = 739, y_bar_nonsmoker = 7.5011, s_nonsmoker = 1.0833
## n_smoker = 107, y_bar_smoker = 7.1713, s_smoker = 0.9724
## (Assuming Zellner-Siow Cauchy prior on the difference of means. )
## (Assuming independent Jeffreys prior on the overall mean and variance. )
## Hypotheses:
## H1: mu_nonsmoker  = mu_smoker
## H2: mu_nonsmoker != mu_smoker
## 
## Priors: P(H1) = 0.5  P(H2) = 0.5 
## 
## Results:
## BF[H2:H1] = 6.237
## P(H1|data) = 0.1382 
## P(H2|data) = 0.8618
\end{verbatim}

Based on the Bayes factor calculated above, how strong is evidence
against \(H_1\)?

\begin{itemize}
\tightlist
\item
  Not worth a bare mention
\item
  Positive
\item
  Strong
\item
  Very Strong
\end{itemize}

How would the Bayes factor above change if we were to increase the prior
probability of \(H_2\) to 0.75? (Hint: you may change the prior of
\(H_1\) and \(H_2\) by specifying
\texttt{hypothesis\_prior\ =\ c(a,\ b)} where \(P(H_1) = a\),
\(P(H_2) = b\), and \(a+b = 1\).)

\begin{itemize}
\tightlist
\item
  Get bigger
\item
  Get smaller
\item
  Stay the same
\end{itemize}

\begin{Shaded}
\begin{Highlighting}[]
\CommentTok{# Type your code for the question here.}
\end{Highlighting}
\end{Shaded}

If differences between the groups are expected to be small, using a
value of \texttt{rscale\ =\ sqrt(2)/2} in the \texttt{JZS} prior is
recommended.

How would the Bayes factor for H2 to H1 change if we were to change the
scale in the Cauchy prior \texttt{rscale\ =\ sqrt(2)/2}?

\begin{itemize}
\tightlist
\item
  Get bigger
\item
  Get smaller
\item
  Stay the same
\end{itemize}

\begin{Shaded}
\begin{Highlighting}[]
\CommentTok{# Type your code for  Question 9 here.}
\end{Highlighting}
\end{Shaded}

To quantify the magnitude of the differences in mean birth weight, we
can use a credible interval. Change the \texttt{type} argument to
\texttt{"ci"} to construct and record a credible interval for the
difference between the weights of babies born to nonsmoking and smoking
mothers, and interpret this interval in context of the data. Note that
by default you'll get a 95\% credible interval. If you want to change
the confidence level, change the value for \texttt{cred\_level} which
takes on a value between 0 and 1. Also note that when doing a credible
interval arguments like \texttt{null} and \texttt{alternative} are not
useful, so make sure to remove them, but include the prior mean
\texttt{mu\_0}.

Based on the 95\% credible interval for the differences in full term
birth weights for nonsmokers and smoker:

\begin{itemize}
\tightlist
\item
  there is a 95\% chance that babies born to nonsmoker mothers are on
  average 0.11 to 0.54 pounds lighter at birth than babies born to
  smoker mothers.
\item
  there is a 95\% chance that the difference in average weights of
  babies whose moms are smokers and nonsmokers is between 0.11 to 0.54
  pounds.
\item
  there is a 95\% chance that the difference in average weights of
  babies in this sample whose moms are nonsmokers and smokers is between
  0.11 to 0.54 pounds.
\item
  there is a 95\% chance that babies born to nonsmoker mothers are on
  average 0.11 to 0.54 pounds heavier at birth than babies born to
  smoker mothers.
\end{itemize}

\begin{Shaded}
\begin{Highlighting}[]
\CommentTok{# Type your code for Question 10 here.}
\end{Highlighting}
\end{Shaded}

\subsection{Bayesian inference on Two Paired
Means}\label{bayesian-inference-on-two-paired-means}

The second data set comes from a 2008 study \emph{A simple tool to
ameliorate detainees' mood and well-being in Prison: Physical
activities}. The study was performed in a penitentiary of the
Rhone-Alpes region (France), that includes two establishments, one for
remand prisoners and short sentences (Jail) and the second for sentenced
persons (Detention Centre, DC). A total number of 26 male subjects,
imprisoned between 3 to 48 months, participated to the study. The
participants were divided into two groups: 15 ``Sportsmen'' who chose
spontaneously to follow the physical program; and 11 ``References'', who
did not and wished to remain sedentary. This data provide the perceived
stress scale (PSS) of the participants in prison at the entry
(\texttt{PSSbefore}) and at the exit (\texttt{PSSafter}).

We can load the \texttt{PrisonStress} data set into our workspace using
the \texttt{data} function once the \texttt{PairedData} package is
loaded.

\begin{Shaded}
\begin{Highlighting}[]
\KeywordTok{data}\NormalTok{(}\StringTok{"PrisonStress"}\NormalTok{)}
\end{Highlighting}
\end{Shaded}

This data set consists of 26 observations on 4 variables. They are
summarized as follows:

\begin{longtable}[]{@{}ll@{}}
\toprule
\begin{minipage}[b]{0.17\columnwidth}\raggedright\strut
variable\strut
\end{minipage} & \begin{minipage}[b]{0.71\columnwidth}\raggedright\strut
description\strut
\end{minipage}\tabularnewline
\midrule
\endhead
\begin{minipage}[t]{0.17\columnwidth}\raggedright\strut
\texttt{Subject}\strut
\end{minipage} & \begin{minipage}[t]{0.71\columnwidth}\raggedright\strut
anonymous subjects\strut
\end{minipage}\tabularnewline
\begin{minipage}[t]{0.17\columnwidth}\raggedright\strut
\texttt{Group}\strut
\end{minipage} & \begin{minipage}[t]{0.71\columnwidth}\raggedright\strut
whether the subject chose to follow the physical programme
\texttt{Sport} or not \texttt{Control}\strut
\end{minipage}\tabularnewline
\begin{minipage}[t]{0.17\columnwidth}\raggedright\strut
\texttt{PSSbefore}\strut
\end{minipage} & \begin{minipage}[t]{0.71\columnwidth}\raggedright\strut
perceived stress measurement at the entry\strut
\end{minipage}\tabularnewline
\begin{minipage}[t]{0.17\columnwidth}\raggedright\strut
\texttt{PSSafter}\strut
\end{minipage} & \begin{minipage}[t]{0.71\columnwidth}\raggedright\strut
perceived stress measurement at the exit\strut
\end{minipage}\tabularnewline
\bottomrule
\end{longtable}

We have two groups of observations: the \texttt{sport} group, the ones
who chose to follow the physical training program; and the
\texttt{control} group, the ones who chose not to follow. We are
interested to know whether in average there is any difference in the
perceived stress scale (PSS) before they started the training (at the
entry) and after the training (at the exit).

We first analyze the \texttt{control} group data. We subset the data
according to the \texttt{Group} variable using the \texttt{dplyr}
package, and save this into a smaller data set \texttt{PPS.control}.

\begin{Shaded}
\begin{Highlighting}[]
\NormalTok{pss_control =}\StringTok{ }\NormalTok{PrisonStress }\OperatorTok
\StringTok{  }\KeywordTok{filter}\NormalTok{(Group }\OperatorTok{==}\StringTok{ "Control"}\NormalTok{) }\OperatorTok
\StringTok{  }\KeywordTok{mutate}\NormalTok{(}\DataTypeTok{diff =}\NormalTok{ PSSbefore }\OperatorTok{-}\StringTok{ }\NormalTok{PSSafter)}
\end{Highlighting}
\end{Shaded}

where the third line calculate the difference of the PSS of each subject
before and after the training and saves it as a new variable
\texttt{diff}.

We can now conduct the following hypothesis test:
\[ H_1: \mu_{\text{before}} = \mu_{\text{after}}\qquad \Longrightarrow \qquad H_1: \mu_{\text{diff}} = 0, \]

\[ H_2: \mu_{\text{before}} \neq \mu_{\text{after}}\qquad \Longrightarrow \qquad H_1: \mu_{\text{diff}} \neq 0, \]

We use \texttt{bayes\_inference} function to calculate the Bayes factor.
The code is similar to the one we used for inference for one mean,
except that we need to set \texttt{null\ =\ 0}, because we are comparing
the mean of the difference to 0.

\begin{Shaded}
\begin{Highlighting}[]
\KeywordTok{bayes_inference}\NormalTok{(}\DataTypeTok{y =}\NormalTok{ diff, }\DataTypeTok{data =}\NormalTok{ pss_control, }
                \DataTypeTok{statistic =} \StringTok{"mean"}\NormalTok{,}
                \DataTypeTok{type =} \StringTok{"ht"}\NormalTok{, }\DataTypeTok{alternative =} \StringTok{"twosided"}\NormalTok{, }\DataTypeTok{null =} \DecValTok{0}\NormalTok{, }
                \DataTypeTok{prior =} \StringTok{"JZS"}\NormalTok{, }\DataTypeTok{rscale =} \DecValTok{1}\NormalTok{, }
                \DataTypeTok{method =} \StringTok{"simulation"}\NormalTok{, }\DataTypeTok{show_plot =} \OtherTok{FALSE}\NormalTok{)}
\end{Highlighting}
\end{Shaded}

\begin{verbatim}
## Single numerical variable
## n = 11, y-bar = -7.3636, s = 9.2333
## (Using Zellner-Siow Cauchy prior:  mu ~ C(0, 1*sigma)
## (Using Jeffreys prior: p(sigma^2) = 1/sigma^2
## 
## Hypotheses:
## H1: mu = 0 versus H2: mu != 0
## Priors:
## P(H1) = 0.5 , P(H2) = 0.5
## Results:
## BF[H2:H1] = 2.7364
## P(H1|data) = 0.2676  P(H2|data) = 0.7324
\end{verbatim}

While there appears to an increase in stress, based on Jeffrey's scales
of evidence, the evidence against H1 is `\emph{worth a bare mention}'.

Conduct the same hypothesis test for the mean of the difference in
perceived stress scale for the \texttt{sport} group. Based of Jeffrey's
scale for interpretation of a Bayes factors how should we describe the
evidence against \(H_1\) from the results?

\begin{itemize}
\tightlist
\item
  Not worth a bare mention
\item
  Positive
\item
  Strong
\item
  Very strong
\end{itemize}

\begin{Shaded}
\begin{Highlighting}[]
\CommentTok{# Type your code for Question 11 here.}
\end{Highlighting}
\end{Shaded}

It is possible that other factors during this period of time could
affect stress. By combining data from both groups we can increase our
sample size, which provides greater power to detect a difference due to
the intervention. If we assume that the variation in the two groups is
comparable, we can use the differences in the before and after
measurements to compare whether the intervention had an effect on stress
levels.

Create a new data frame with a variable that is the difference in pre
and post stress measurements and test the hypothesis that the mean
difference in the control group is equal to the mean in the sport group
versus the hypothesis that the means are not equal.

\begin{Shaded}
\begin{Highlighting}[]
\CommentTok{# Type your code for Exercise 2 here.}
\end{Highlighting}
\end{Shaded}


\end{document}
